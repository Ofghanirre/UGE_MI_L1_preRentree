\documentclass{article}

\def\titre{Session 3 : Fondamentaux informatique pour bien démarrer les cours}

\usepackage[a4paper,left=20mm,right=20mm,top=25mm,bottom=40mm]{geometry}
\usepackage{enumitem}
\setlist{partopsep=0pt,topsep=1ex}
\usepackage{pifont,euscript}
\usepackage{rotating}
\usepackage{tabu}
\usepackage[usenames,dvipsnames,svgnames]{xcolor}
\usepackage{comment}
\usepackage{colortbl}
\usepackage{titling}
\usepackage{multicol}
\usepackage[french]{babel}
\usepackage{mdframed}
\usepackage{textcomp}
\usepackage{listings}
\lstloadlanguages{Python}

\setlength{\droptitle}{-1.5cm}
\pretitle{\begin{center}\LARGE\sffamily\bfseries}
\posttitle{\par\end{center}\vskip -.5cm}
\preauthor{}
\postauthor{}
\predate{}
\postdate{}

\definecolor{UGEBlue}{RGB}{47,42,133}
\definecolor{vert}{RGB}{78,100,26}
\definecolor{violet}{RGB}{130,25,111}
\definecolor{orange}{RGB}{234,91,12}
\definecolor{rouge}{RGB}{230,56,18}
\definecolor{beige}{RGB}{242,238,231}
\definecolor{bleu}{RGB}{0,74,155}
\definecolor{gris}{RGB}{100,100,100}
\definecolor{commentColor}{rgb}{0.38,0.63,0.69}
\definecolor{stringColor}{rgb}{0.25,0.44,0.63}
\definecolor{typeColor}{rgb}{0.56,0.13,0.00}
\definecolor{keywordColor}{rgb}{0.00,0.44,0.13}
\definecolor{digitColor}{rgb}{0.25,0.63,0.44}

\lstset{ %
  backgroundcolor=\color{white},
  basicstyle=\ttfamily,
  breaklines=true,
  captionpos=b,
  commentstyle={\bfseries\itshape\color{commentColor}},
  escapeinside={\%*}{*)},
  keywordstyle=\color{keywordColor},
  stringstyle=\color{stringColor},
  numbers=left,
  stepnumber=1,
  numberstyle=\scriptsize,
  numbersep=10pt,
  language=Python,
  upquote=true
}
\lstset{%
  literate=%
    {é}{{\'e}}1
    {è}{{\`e}}1
    {ê}{{\^e}}1
    {ë}{{\"e}}1
    {à}{{\`a}}1
    {â}{{\^a}}1
    {ä}{{\"a}}1
    {î}{{\^i}}1
    {ï}{{\"i}}1
    {ù}{{\`u}}1
    {û}{{\^u}}1
    {ü}{{\"u}}1
    {ô}{{\^o}}1
    {ö}{{\"o}}1
    {ç}{{\c c}}1
    {œ}{{\oe{}}}1
    {0}{{{{\color{digitColor}0}}}}1
    {1}{{{{\color{digitColor}1}}}}1
    {2}{{{{\color{digitColor}2}}}}1
    {3}{{{{\color{digitColor}3}}}}1
    {4}{{{{\color{digitColor}4}}}}1
    {5}{{{{\color{digitColor}5}}}}1
    {6}{{{{\color{digitColor}6}}}}1
    {7}{{{{\color{digitColor}7}}}}1
    {8}{{{{\color{digitColor}8}}}}1
    {9}{{{{\color{digitColor}9}}}}1
    {.0}{{{{\color{digitColor}.0}}}}2
    {.1}{{{{\color{digitColor}.1}}}}2
    {.2}{{{{\color{digitColor}.2}}}}2
    {.3}{{{{\color{digitColor}.3}}}}2
    {.4}{{{{\color{digitColor}.4}}}}2
    {.5}{{{{\color{digitColor}.5}}}}2
    {.6}{{{{\color{digitColor}.6}}}}2
    {.7}{{{{\color{digitColor}.7}}}}2
    {.8}{{{{\color{digitColor}.8}}}}2
    {.9}{{{{\color{digitColor}.9}}}}2
    {'h}{{{{\color{digitColor}'h}}}}2
}
\lstset{%
  emph={int,char,void,float,double},
  emphstyle=\color{typeColor}
}

\usepackage{lmodern}
\usepackage{amssymb,amsmath}
\usepackage{ifxetex,ifluatex}
\usepackage[T1]{fontenc}
\usepackage[utf8]{inputenc}
\usepackage{upquote}

\usepackage[unicode=true]{hyperref}
\hypersetup{breaklinks=true,
%            bookmarks=true,
            pdfauthor={},
            pdftitle={\titre},
            colorlinks=true,
            citecolor=blue,
            urlcolor=blue,
            linkcolor=magenta,
            pdfborder={0 0 0}}
\urlstyle{same}  % don't use monospace font for urls
\usepackage{tikz}
\usetikzlibrary{quotes}
\usetikzlibrary{arrows,positioning} 

\tikzstyle{val}=[rectangle,draw]
\tikzstyle{var}=[rectangle,rounded corners=1.5ex,draw,bleu,minimum size=3ex]
\tikzstyle{aff}=[->,thick]
\tikzset{
  every node/.style={font=\footnotesize},
  instr/.style={rectangle,draw=black!70},
  test/.style={rectangle,rounded corners,draw=bleu!70,text=bleu},
  every edge/.style={>=stealth,semithick,bend angle=45,draw,->},
  every edge quotes/.style={inner sep=.5em,font=\scriptsize,color=orange},
  vrai/.style={left,"vrai"},
  faux/.style={right,"faux"},
}
\usepackage{color}
\usepackage{fancyvrb}
\setlength{\parindent}{0pt}
\setlength{\parskip}{6pt plus 2pt minus 1pt}
\setlength{\emergencystretch}{3em}  % prevent overfull lines
\providecommand{\tightlist}{%
  \setlength{\itemsep}{0pt}\setlength{\parskip}{0pt}}
\setcounter{secnumdepth}{5}

\usepackage{fancyhdr}
\fancyhead[L]{\includegraphics[height=1.1cm]{UPEM-IGM-V1_300dpi.png}}
\fancyhead[C]{{
\bf Pré-rentrée informatique\\
Semaine d'intégration du 5 au 9 septembre 2022\\
}
L1 Informatique \& Mathématiques
}
%\fancyhead[R]{\includegraphics[height=1.3cm]{logo-ligm.png}}
\setlength{\headheight}{50pt}
\renewcommand{\headrulewidth}{1pt}


% cadres informatifs
\newcommand{\but}[1]{%
\hbox{}\noindent%
\fcolorbox{CadetBlue}{white}{%
  \parbox{\dimexpr\linewidth-2\fboxsep-1pt\relax}{%
    \vskip10pt%
    \leftskip10pt\rightskip10pt%
    #1%
    \vskip10pt}}\medskip\thispagestyle{fancy}}


% Environnements d'exercices

\font\manual=manfnt % font used for the METAFONT logo, etc.
\def\barble{{\manual\char1}}
\def\fulltriangleright{{\manual\char120}} %
\def\fulltriangleup{{\manual\char54}} %
\def\fulltriangledown{{\manual\char55}}
\def\littlecube{{\manual\char28}}
\def\prettystar{{\manual\char30}}
\def\littlecircles{{\manual\char36}}
\def\littlelosanges{{\manual\char37}}
\def\littlecloud{{\manual\char38}}
\def\ellipsoid{{\manual\char88}}
\def\zip{{\manual\char121}} % -/\/->
\def\dbendright{{\manual\char126}}
\def\dbendleft{{\manual\char127}}
\def\question{\noindent\llap{\ding{90}\ding{212}}{\kern1pc}}
\newcounter{exo_count}
\def\exoname{Exercice}
\newenvironment{exercice}[1][]%
{\smallbreak\refstepcounter{exo_count}%
\vspace{0.3cm}%
\begin{sloppypar}\noindent{\fulltriangleright\kern 2pt%
\bf\exoname~\arabic{exo_count} : #1} %
}%
{\end{sloppypar}%
}



\newenvironment{correction}{\subsection*{\textcolor{red}{Correction}}}{\medskip}

\def\labelenumi{\arabic{enumi}.}
\def\labelenumii{\alph{enumii}.}

\title{\color{UGEBlue}\titre}
%\date{Semestre 2}
\date{}


\newcommand{\code}[1]{\lstinline{#1}}
\lstnewenvironment{fullcode}{\lstset{xleftmargin=1.4em}}{}
\lstnewenvironment{nonumbercode}{\lstset{numbers=none}}{}


\excludecomment{correction}
% Commenter la ligne ci-dessus si on souhaite afficher les corrections

\begin{document}
\maketitle

\but{Dans ce TP vous allez:
\begin{itemize}
\item Manipuler l'IDE Thonny ainsi que PythonTutor qui seront utilisés pour les cours de Python (AP1/APP1);
\item ...
 \end{itemize} 
}

\section {Bien suivre les cours et se former à Python}

Vous allez bientôt commencer votre formation en L1 ainsi que vos cours en algorithmique et programmation avec le langage Python. Ce chapitre est dédié à la présentation des outils et applications qui vous permettront de bien démarrer.

\subsection{Thonny : éditeur Python pour les cours d'AP1 (APP1 et autre ...)}

Quand on découvre un langage de programmation et surtout la logique algorithmique, il faut utiliser des outils simples, facile d'accès et de compréhension.

À la vérité quand on débute en Python, un simple éditeur de texte est très suffisant, comme vous avez pu le voir à l'exercice 9. 

Cependant, plus on progresse dans les subtilités du langage et plus on a envie d'aller un peu plus ``vite'' et de bénéficier d'un éditeur de texte plus convivial, plus riche de fonctionnalités d'assistance comme par exemple la complétion de code ou la coloration syntaxique du langage (Il s'agit d'un formatage automatique du texte avec une couleur et une fonte spécifique en fonction de son type : variable, fonction, ...). \\
Bref on parle ici d'IDE (Integrated Development Environment), soit en fran\c cais; Environnement Intégré de Développement.

Sur internet, dans votre entourage et auprès de vos collègues des années précédentes vous trouverez et entendrez parler d'une multitude d'IDEs tous plus pointus et complets les uns les autres. Néanmoins, plus pointu et plus complet veut dire aussi plus complexe voir beaucoup plus complexe et il y a de fortes chances pas très utile pour le niveau d'une L1 (qu'on soit en MIASHS, AP1 ou en APP1) ! \\
Des IDEs comme PyCharm, Visual Studio Code (VS Code) ou Spyder peuvent être très séduisants et ont pignon sur internet (\c ca veut dire qu'internet ne jurent que par eux). Ces IDEs font littéralement le café ... à la condition de savoir s'en servir ... à la condition d'en avoir réellement besoin aussi. \\
Pour résumer, un IDE aussi fantastique soit-il ne codera jamais pour vous ! Il s'agit avant tout d'un assistant éditeur ``intelligent'' pour faciliter la vie des développeurs. 

Soyons clair, personne ne vous empêchera de choisir l'IDE que vous voudrez tant que vous faites, avec ce dernier, ce qui est demandé pendant vos cours.

Pour vous faciliter la vie, l'équipe enseignante d'AP1 vous propose l'IDE Thonny : \url{https://thonny.org/}. Thonny est un IDE très simple et très adapté à vos premiers pas sur Python. 

Dans cette section, l'objectif n'est pas de vous former à Thonny mais de vous donner les moyens de le faire et éventuellement de l'installer sur vos machines perso. Il y a d'ailleurs un tuto sympa en fran\c cais que vous pourriez suivre pour vous former : \url{https://www.codeflow.site/fr/article/python-thonny}.

\subsection{Python Tutor}

Dans la série des outils qui vous aideront à survivre à vos cours de L1 et potentiellement même après (avec d'autres langages), 
Python Tutor est un must (\url{https://pythontutor.com/}) ! 

Cet outil permet de visualiser en live le déroulement d'un programme ligne par ligne en présentant les variables dans un graphique très intuitif. Il a aussi l'avantage de ne pas nécessiter d'installation. 

Python Tutor est un outil pédagogique qui est très largement utilisé par les enseignants. Savoir l'utiliser serait un plus. 

À ce sujet regardez la vidéo sous Elearning : \emph{``PythonTutor.mov''}.

\textbf{ATTENTION : Il ne s'agit pas d'un IDE, vous ne pourrez pas programmer avec ! Vous ne pourrez pas sauvegarder votre code naturellement non plus ... puisque ce n'est pas un IDE.}


Voilà c'est la fin de la Session\_3 ! 

A présent vous êtes formés et préparés pour entrer de plein pied dans votre formation ! 

Nous vous souhaitons une pleine et complète réussite.

\end{document}